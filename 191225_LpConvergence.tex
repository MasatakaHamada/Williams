\documentclass{jsarticle}
\usepackage{amsmath, amssymb, amsthm}
\newtheorem{thm}{定理}

\begin{document}

以下,$p\geq1$とする.

\begin{thm}\label{有界収束定理}
確率変数列$(X_n)$が確率変数$X$に確率収束するとする.
さらに,ある定数$K\in[0,\infty)$が存在して,任意の$n,\omega$に対して
$|X_n(\omega)|\leq K$が成り立つとする.
このとき,$X_n$は$X$に$L^p$-収束する.
\end{thm}

\begin{proof}
$P(|X|\leq K)=1$であることを確かめよう.
いま,$k\geq1$に対して
\[
|X|>K+k^{-1}
\Rightarrow|X-X_n|\geq|X|-|X_n|
>(K+k^{-1})-K=k^{-1}
\]
であるから,任意の$n$に対して
$P(|X|>K+k^{-1})\leq P(|X-X_n|>k^{-1})$
である.
ここで,$X_n$は$X$に確率収束するから,この不等式の右辺は0に収束する.
よって$P(|X|>K+k^{-1})=0$である.
従って
\[
P(|X|>K)=P\left(\bigcup_{k\geq1}\{|X|>K+k^{-1}\}\right)
\leq\sum_{k\geq1}P(|X|>K+k^{-1})=0
\]
ゆえ$P(|X|\leq K)=1$である.

$\varepsilon>0$とする.$\varepsilon$に対して次を満たすような$n_0$をとる.
\[
n\geq n_0\Rightarrow P(|X_n-X|>\varepsilon/3)<\left(\frac{\varepsilon}{3K}\right)^p
\]
このとき,$n\geq n_0$に対して
\begin{align*}
\|X_n-X\|_p
&=\{E[|X_n-X|^p:|X_n-X|>\varepsilon/3]+E[|X_n-X|^p:|X_n-X|\leq\varepsilon/3]\}^{1/p} \\
&\leq\{E[|X_n-X|^p:|X_n-X|>\varepsilon/3]+(\varepsilon/3)^p\}^{1/p} \\
&\leq\{E[(|X_n|+|X|)^p:|X_n-X|>\varepsilon/3]+(\varepsilon/3)^p\}^{1/p} \\
&\leq\{(2K)^pP(|X_n-X|>\varepsilon/3)+(\varepsilon/3)^p\}^{1/p} \\
&\leq\{(2K)^p(\varepsilon/3K)^p+(\varepsilon/3)^p\}^{1/p}
=\{(2\varepsilon/3)^p+(\varepsilon/3)^p\}^{1/p} \\
&\leq\{(2\varepsilon/3+\varepsilon/3)^p\}^{1/p}=\varepsilon
\end{align*}
となる.従って,$X_n$は$X$に$L^p$-収束する.
\end{proof}

\begin{thm}
$(X_n)$を$L^p$の元の列,$X\in L^p$とする.
このとき,$X_n$が$X$に$L^p$-収束するための必要十分条件は
\begin{enumerate}
\item$X_n$は$X$に確率収束する,
\item$(|X_n|^p)$は一様可積分である
\end{enumerate}
がともに成り立つことである.
\end{thm}

\begin{proof}
1と2がともに満たされているとしよう.
$K\in[0,\infty)$に対して,関数$\varphi_K\colon\mathbb{R}\to[-K,K]$を次のように定める.
\[
\varphi_K(x)=
\begin{cases}
-K & (x<-K) \\
x & (|x|\leq K) \\
K & (x>K)
\end{cases}
\]
$\varepsilon>0$とする.
\begin{align*}
\|\varphi_K(X)-X\|_p
&=\{E[|K-X|^p:X>K]+E[|-K-X|^p:X<-K]\}^{1/p} \\
&\leq\{E[(K+|X|)^p:|X|>K]\}^{1/p} \\
&\leq2\{E[|X|^p:|X|>K]\}^{1/p}
\end{align*}
であり,
13.1(b)より$E[|X|^p:|X|>K_1]<(\varepsilon/6)^p$を満たす$K_1\in[0,\infty)$が存在する.
同様に
\[
\|\varphi_K(X_n)-X_n\|_p\leq2\{E[|X_n|^p:|X_n|>K]\}^{1/p}
\]
であり,$(|X_n|^p)$の一様可積分性から,ある$K_2\in[0,\infty)$が存在して,
任意の$n$に対して$E[|X_n|^p:|X_n|>K_2]<(\varepsilon/6)^p$を満たす.
そこで,$K=\max\{K_1,K_2\}$とおけば,任意の$n$に対して
\[
\|\varphi_K(X)-X\|_p\leq\frac{\varepsilon}{3},
\|\varphi_K(X_n)-X_n\|_p\leq\frac{\varepsilon}{3}
\]
が成り立つ.
さらに,$|\varphi_K(x)-\varphi_K(y)|\leq|x-y|$であることに注意すると,
$X_n$が$X$に確率収束するので,$\varphi_K(X_n)$も$\varphi_K(X)$に確率収束する.
よって,定理\ref{有界収束定理}より次を満たす$n_0$が存在する.
\[
n\geq n_0
\Rightarrow
\|\varphi_K(X_n)-\varphi_K(X)\|_p<\frac{\varepsilon}{3}
\]
従って,$n\geq n_0$に対して
\[
\|X_n-X\|_p
\leq\|\varphi_K(X)-X\|_p+\|\varphi_K(X_n)-\varphi_K(X)\|_p+\|\varphi_K(X_n)-X_n\|_p
<\frac{\varepsilon}{3}+\frac{\varepsilon}{3}+\frac{\varepsilon}{3}
=\varepsilon
\]
であり,$X_n$が$X$に$L^p$-収束することがわかる.

逆に,$X_n$が$X$に$L^p$-収束するとしよう.
$\varepsilon>0$に対して
\[
n\geq N\Rightarrow\|X_n-X\|_p<\frac{\varepsilon^{1/p}}{2}
\]
となるような$N$を選ぶ.13.1(a)より
\begin{itemize}
\item 各$1\leq n\leq N$に対して$P(F)<\delta_n\Rightarrow E(|X_n|^p;F)<\varepsilon$
\item $P(F)<\delta'\Rightarrow E(|X|^p;F)<\varepsilon/2$
\end{itemize}
を満たす$\delta_1,\cdots,\delta_N,\delta'>0$が存在する.
そこで,$\delta=\min\{\delta_1,\cdots,\delta_N,\delta'\}$とおけば,
各$1\leq n\leq N$に対して
\[
P(F)<\delta\Rightarrow E(|X_n|^p;F)<\varepsilon,
P(F)<\delta\Rightarrow E(|X|^p;F)<\varepsilon/2^p
\]
が成り立つ.
収束列が有界であることに注意すると,$X_n$が$X$に$L^p$-収束することから,
$(X_n)$は$L^p$-有界である.よって,$\sup_n\|X_n\|_p<(K\delta)^{1/p}$を満たす$K>0$が存在する.
このとき
\[
P(|X_n|^p>K)=K^{-1}E(K:|X_n|^p>K)
\leq K^{-1}E(|X_n|^p:|X_n|^p>K)
\leq K^{-1}\|X_n\|_p^p<\delta
\]
であるから,$n\geq N$に対して
\begin{align*}
E(|X_n|^p;|X_n|^p>K)
&=\|X_n1_{\{|X_n|^p>K\}}\|_p^p \\
&\leq(\|(X_n-X)1_{\{|X_n|^p>K\}}\|_p+\|X1_{\{|X_n|^p>K\}}\|_p)^p \\
&=(E[|X_n-X|^p:|X_n|^p>K]^{1/p}+E[|X|^p:|X_n|^p>K]^{1/p})^p \\
&\leq(\|X_n-X\|_p+(\varepsilon/2^p)^{1/p})^p \\
&\leq(\varepsilon^{1/p}/2+\varepsilon^{1/p}/2)^p=\varepsilon
\end{align*}
となる.また,$1\leq n\leq N$に対しても$E(|X_n|^p;|X_n|^p>K)<\varepsilon$である.
従って,$(|X_n|^p)$は一様可積分である.
一方
\[
P(|X_n-X|>\varepsilon)
=\frac{E[\varepsilon;|X_n-X|>\varepsilon]}{\varepsilon}
\leq\frac{E[|X_n-X|;|X_n-X|>\varepsilon]}{\varepsilon}
\leq\frac{\|X_n-X\|_1}{\varepsilon}
\leq\frac{\|X_n-X\|_p}{\varepsilon}
\]
より,$X_n$が$X$に確率収束することもわかる.
\end{proof}

\end{document}
