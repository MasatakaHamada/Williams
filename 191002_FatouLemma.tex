\documentclass{jsarticle}
\usepackage{amsmath, amssymb, amsthm}
\newtheorem{thm}{定理}
\newtheorem{cor}{系}

\begin{document}
$(\Omega,\mathcal{F},P)$を確率空間とする.

\begin{thm}[ファトゥの補題]
$(\Omega,\mathcal{F},P)$上の非負確率変数列$\{X_n:n\in\mathbb{N}\}$に対して
次の不等式が成り立つ.
\[ E\left(\liminf_nX_n\right)\leq\liminf_nE(X_n) \]
\end{thm}

\begin{cor}
$\{X_n:n\in\mathbb{N}\}$を$(\Omega,\mathcal{F},P)$上の確率変数列とする.
このとき,ある可積分確率変数$Y$が存在して,任意の$n\in\mathbb{N}$に対して
$X_n\geq Y$であるとき,次の不等式が成り立つ.
\[ E\left(\liminf_nX_n\right)\leq\liminf_nE(X_n) \]
\end{cor}
\begin{proof}
$Z_n=X_n-Y$とおくと,仮定より$\{Z_n:n\in\mathbb{N}\}$は非負確率変数列である.
よって,ファトゥの補題より
\begin{equation}\label{byfatou}
E\left(\liminf_nZ_n\right)\leq\liminf_nE(Z_n)
\end{equation}
が成り立つ.ここで
\begin{align*}
(\text{(\ref{byfatou})の左辺})
&=E\left(\liminf_n(X_n-Y)\right)
=E\left(\liminf_nX_n-Y\right)
=E\left(\liminf_nX_n\right)-E(Y),\\
(\text{(\ref{byfatou})の右辺})
&=\liminf_nE(X_n-Y)
=\liminf_n\{E(X_n)-E(Y)\}
=\liminf_nE(X_n)-E(Y)
\end{align*}
であるから,(\ref{byfatou})より
\[ E\left(\liminf_nX_n\right)\leq\liminf_nE(X_n) \]
が得られる.
\end{proof}
\end{document}
