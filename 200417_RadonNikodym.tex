\documentclass{jsarticle}
\usepackage{amsmath,amssymb,amsthm}
\theoremstyle{definition}
\newtheorem{lem}{Lemma}

\begin{document}
% https://math.stackexchange.com/questions/2558790/radon-nikodym-theorem-for-positive-measures-chain-rule/
\begin{lem}
$(S,\Sigma,\mu)$を測度空間とし,$f$を$(S,\Sigma)$上の非負可測関数とする.
また,$(S,\Sigma)$上の測度$\nu$を$\nu(A)=\int_Af\,d\mu,A\in\Sigma$により定める.
このとき,$(S,\Sigma)$上の任意の非負可測関数$F$に対して次の等式が成り立つ:
\[ \int_SF\,d\nu=\int_SFf\,d\mu.\]
\end{lem}
\begin{proof}
$F=I_A,A\in\Sigma$のとき
\[ \int_SF\,d\nu=\int_SI_A\,d\nu=\nu(A)=\int_Af\,d\mu=\int_SI_Af\,d\mu=\int_SFf\,d\mu \]
であり,等式は成り立つ.$F$が非負単関数であるとき,$F=\sum_{i=1}^na_iI_{A_i}$と書くと,
積分の線形性より
\[
\int_SF\,d\nu=\sum_{i=1}^na_i\int_SI_{A_i}\,d\nu=\sum_{i=1}^na_i\int_SI_{A_i}f\,d\mu
=\int_SFf\,d\mu
\]
であり,等式は成り立つ.ここで,各$n\in\mathbb{N}$に対し,$a^{(n)}\colon[0,\infty]\to[0,n]$を
\[
a^{(n)}(x)=\begin{cases}0 & (x=0) \\
(i-1)2^{-n} & ((i-1)2^{-n}<x\leq i2^{-n}\leq n,i\in\mathbb{N}) \\
n & (x>n)\end{cases}
\]
で定める.$F$が一般の非負可測関数であるとき,$n\in\mathbb{N}$に対し$F_n=a^{(n)}\circ F$とおくと,
$(F_n)$は$0\leq F_n\uparrow F$を満たす非負単関数列である.よって,単調収束定理より
\[
\int_SF\,d\nu=\lim_{n\to\infty}\int_SF_n\,d\nu=\lim_{n\to\infty}\int_SF_nf\,d\mu
=\int_SFf\,d\mu
\]
であり,等式は成り立つ.
\end{proof}
\begin{lem}
$X=\frac{dQ}{dP}$のとき,$P(X>0)=1$ならば$X^{-1}=\frac{dP}{dQ}$である.
\end{lem}
\begin{proof}
任意の$F\in\Sigma$に対して,上記の補題より
\[ \int_FX^{-1}\,dQ=\int_FX^{-1}X\,dP=\int_FdP=P(F) \]
が成り立つ.従って$X^{-1}=\frac{dP}{dQ}$である.
\end{proof}
\end{document}
