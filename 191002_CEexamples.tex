\documentclass{jsarticle}
\usepackage{amsmath, amssymb}

\begin{document}
\section*{初等的な条件付き期待値の具体例}

$\Omega=\{1,2,3,4,5,6\}$,$\mathcal{F}=2^\Omega$とし,
$(\Omega,\mathcal{F})$上の確率測度$P$を
$P(\{\omega\})=\dfrac{1}{6}$,$\omega\in\Omega$
により定める.

$(\Omega,\mathcal{F},P)$上の確率変数$X,Z$を
\[ X(\omega)=\omega,Z(\omega)=(\omega\text{を3で割った余り}) \]
で定義する.このとき
\[ P(X=1\mid Z=1)=\frac{P(X=1;Z=1)}{P(Z=1)}=\frac{1/6}{1/3}=\frac{1}{2},
同様に\ P(X=4\mid Z=1)=\frac{1}{2} \]
である.
また,
\[ P(X=2\mid Z=1)=P(X=3\mid Z=1)=P(X=5\mid Z=1)=P(X=6\mid Z=1)=0 \]
である.よって
\[ E(X\mid Z=1)=\sum_{\omega=1}^6\omega P(X=\omega\mid Z=1)
=1\cdot\frac{1}{2}+4\cdot\frac{1}{2}=\frac{5}{2} \]
である.同様にして
\[ E(X\mid Z=0)=3\cdot\frac{1}{2}+6\cdot\frac{1}{2}=\frac{9}{2},
E(X\mid Z=2)=2\cdot\frac{1}{2}+5\cdot\frac{1}{2}=\frac{7}{2} \]
となる.
ここで,$(\Omega,\mathcal{F},P)$上の確率変数$Y$を
\[ Y(\omega)=\begin{cases}9/2 & (Z(\omega)=0\text{ i.e. }\omega=3,6) \\
5/2 & (Z(\omega)=1\text{ i.e. }\omega=1,4) \\
7/2 & (Z(\omega)=2\text{ i.e. }\omega=2,5)\end{cases} \]
と定め,$Y$を$X$の$Z$に関する条件付き期待値といい,$Y=E(X\mid Z)$と書く.
\end{document}
